% Options for packages loaded elsewhere
\PassOptionsToPackage{unicode}{hyperref}
\PassOptionsToPackage{hyphens}{url}
%
\documentclass[
]{article}
\usepackage{amsmath,amssymb}
\usepackage{lmodern}
\usepackage{ifxetex,ifluatex}
\ifnum 0\ifxetex 1\fi\ifluatex 1\fi=0 % if pdftex
  \usepackage[T1]{fontenc}
  \usepackage[utf8]{inputenc}
  \usepackage{textcomp} % provide euro and other symbols
\else % if luatex or xetex
  \usepackage{unicode-math}
  \defaultfontfeatures{Scale=MatchLowercase}
  \defaultfontfeatures[\rmfamily]{Ligatures=TeX,Scale=1}
\fi
% Use upquote if available, for straight quotes in verbatim environments
\IfFileExists{upquote.sty}{\usepackage{upquote}}{}
\IfFileExists{microtype.sty}{% use microtype if available
  \usepackage[]{microtype}
  \UseMicrotypeSet[protrusion]{basicmath} % disable protrusion for tt fonts
}{}
\makeatletter
\@ifundefined{KOMAClassName}{% if non-KOMA class
  \IfFileExists{parskip.sty}{%
    \usepackage{parskip}
  }{% else
    \setlength{\parindent}{0pt}
    \setlength{\parskip}{6pt plus 2pt minus 1pt}}
}{% if KOMA class
  \KOMAoptions{parskip=half}}
\makeatother
\usepackage{xcolor}
\IfFileExists{xurl.sty}{\usepackage{xurl}}{} % add URL line breaks if available
\IfFileExists{bookmark.sty}{\usepackage{bookmark}}{\usepackage{hyperref}}
\hypersetup{
  pdftitle={ENSC 445/545 Watershed Analysis},
  pdfauthor={null},
  hidelinks,
  pdfcreator={LaTeX via pandoc}}
\urlstyle{same} % disable monospaced font for URLs
\usepackage[margin=1in]{geometry}
\usepackage{graphicx}
\makeatletter
\def\maxwidth{\ifdim\Gin@nat@width>\linewidth\linewidth\else\Gin@nat@width\fi}
\def\maxheight{\ifdim\Gin@nat@height>\textheight\textheight\else\Gin@nat@height\fi}
\makeatother
% Scale images if necessary, so that they will not overflow the page
% margins by default, and it is still possible to overwrite the defaults
% using explicit options in \includegraphics[width, height, ...]{}
\setkeys{Gin}{width=\maxwidth,height=\maxheight,keepaspectratio}
% Set default figure placement to htbp
\makeatletter
\def\fps@figure{htbp}
\makeatother
\setlength{\emergencystretch}{3em} % prevent overfull lines
\providecommand{\tightlist}{%
  \setlength{\itemsep}{0pt}\setlength{\parskip}{0pt}}
\setcounter{secnumdepth}{-\maxdimen} % remove section numbering
\ifluatex
  \usepackage{selnolig}  % disable illegal ligatures
\fi

\title{ENSC 445/545 Watershed Analysis}
\author{null}
\date{null}

\begin{document}
\maketitle

\textbf{Instructor}: Dr.~Tim Covino

\textbf{Class times}: T 14:00 -- 15:15; Th 14:00 -- 15:15

\textbf{Office hours}: W 14:00 - 15:15; or by appointment

\textbf{Website}:
\url{https://tpcovino.github.io/Git_home/watershed_analysis/index.html}

\textbf{Course overview and objectives}: (1) provide theoretical
understanding and practical experience with the most common analysis and
modeling techniques relevant to watershed hydrology; and (2) provide
training in analyzing, simulating, and presenting scientific data in
written and oral formats.

\textbf{Approach}: This class will be largely hands-on, and students
will be conducting watershed analyses and modeling exercises. We will
follow a typical weekly format of Tuesday in-class lecture and
activities, Thursday hands-on computational work, and time spent working
through online modules outside of class. The online modules are required
and are intended to prepare students for the in-class work during lab on
Thursday. Accordingly, this class will follow a semi-flipped format that
includes in-class lecture, online modules, and in-class work sessions.

\textbf{Course structure}: Tuesday lectures will provide theoretical
understanding and Thursday computing labs will be focused on skill
development in watershed analysis and modeling.

\textbf{Tentative schedule, subject to change}:

Week 1: Introduction, overview and technical skills.

Week 2: Watershed boundaries, delineation and terrain analysis.

Week 3: Hydrologic processes, climate and water balance. Trend detection
and analysis, non-parametric approaches.

Week 4: Statistical vs process-based models and their implementation.

Week 5: Surface water: Rating curves, hydrographs, frequency analysis
and modeling approaches.

Week 6: Precipitation variability, interpolation schemes and data
sources.

Week 7: Evapotranspiration, physical processes, and modeling approaches.

Week 8: Spring break.

Week 9: Term project overview, project brainstorm, identify data/models
necessary to complete project.

Week 10: Groundwater, physical processes, data availability, and
modeling approaches.

Week 11: Erosional processes and modeling.

Week 12: Water quality: analysis and prediction.

Week 13: Land cover change, remote sensing, and implications for
watershed hydrology.

Week 14: Writing as rewriting.

Week 15: Compelling presentations.

Week 16: Final presentations.

\textbf{Grading}: Online module assignments: 25\% Weekly lab
assignments: 50\% Final presentation: 10\% Final report: 15\%

\end{document}
